\section{Related Work}

Reachable program behaviors (e.g. code coverage and discovered bugs) determine the \emph{effectiveness} of a software testing process. For non-trivial programs, exhaustively exercising all possible behaviors is infeasible. Consequently, testing is often viewed as a means to demonstrate the presence of bugs, but not their absence~\cite{dijkstra2022reliability}. Despite this limitation, substantial effort has been directed toward increasing maximum reachability to enhance software security and reliability. Nevertheless, determining reachability remains a fundamental challenge in software testing. In fact, an efficient and precise determination of reachable code would effectively resolve the software verification problem~\cite{reachability_2023}. 

Efforts to approximate reachability quantitatively have gained traction in the security domain, despite inherent difficulties. Empirical studies indicate that establishing the reachability of certain code regions is particularly challenging in large, complex code bases, using both static and dynamic analysis techniques~\cite{reachability_questions}. Static analysis methods, such as symbolic execution, often suffer from over- or under-approximation when estimating reachable coverage~\cite{reachability_2023}. For instance, Nikoli\'{c} and Spoto~\cite{reachability_program_vars} proposed approximating the reachability of program variables as a new abstract domain for static analysis. While their approach yields over-approximations, authors argue that it can be conservatively applied to identify unreachable code. Similarly, Mikol\'{a}\v{s} et al.~\cite{annotate_reachability} leveraged annotated code to define unreachability conditions and proposed an efficient algorithm for detecting unreachable code.

Assessing the reachability or unreachability of dynamic analysis also became popular in recent times. Naus et al. \cite{naus2023low} attempted in finding inputs that trigger a post-condition (a bug) through automatically generating preconditions for reachability using low-level code. A recent work that explores statistical methods to estimate reachability probability and highlights the need of structure-aware estimators opposed to the ones that are not aware of the internal program structure \cite{statreachability_2023}.


\todo{Killable mutatants estimation \cite{killable}}