\lstdefinestyle{Python}
{
    basicstyle=\footnotesize\ttfamily,
    numberblanklines=false,
    language=python,
    tabsize=2,
    commentstyle=\color{gray},
    keywordstyle=\bfseries\color{eclipsePurple},
    morekeywords={assert},
    stringstyle=\color{eclipseBlue},
    procnamestyle=\bfseries\color{black},
    procnamekeys={def},
    columns=flexible,
    identifierstyle=,
}
\begin{figure} %{r}{7cm} %[12]
\begin{subfigure}[h]{0.24\textwidth} %22 min
%\begin{subfigure}
\lstset{numbers=left,xleftmargin=2em,
numberstyle=\color{lightgray}
} %frame=single,framexleftmargin=1.5em}
\begin{lstlisting}[style=Python, escapechar=|,numbersep=2pt]
def bsearch(x, v, n):
  low, high = 0, n-1
  while low <= high:
    mid=(low+high)/2
    if x < v[mid]:
      high=mid-1
    else:
      if x > v[mid]:
        low=mid+1
      else:
        return mid
  return None
\end{lstlisting}
\vspace{60pt}
\end{subfigure}
\begin{subfigure}[h]{0.29\textwidth} %25 min
%\begin{subfigure}
\lstset{numbers=left,xleftmargin=2em,
numberstyle=\color{lightgray}
} %frame=single,framexleftmargin=1.5em}
\begin{lstlisting}[style=Python, escapechar=|,numbersep=2pt]
def bsearch(x, v, n):
  low, high = 0, n-1
  while True:
    r1_ = low <= high
    if r1_:
      mid = (low+high)/2
      r2_ = x < v[mid]
      if r2_:
        high = mid-1
      else:
        r3_ = x > v[mid]
        if r3_:
          low = mid+1
        else:
          return mid
    else:
      break
  return None
\end{lstlisting}
\end{subfigure}
\begin{subfigure}[h]{0.28\textwidth}   %28 min
\begin{grammar}%\centering
  <bsearch> $\rightarrow$ l.2 <while.3> l.18

  <while.3> $\rightarrow$ $\epsilon$
   \alt l.4 <if.5> <while.3>

  <if.5>    $\rightarrow$ l.6 l.7 <if.8>
   \alt l.17

  <if.8> $\rightarrow$ l.9
   \alt l.11 <if.12>

  <if.12> $\rightarrow$ l.13
    \alt l.15
\end{grammar}
\vspace{10pt}
\end{subfigure}
\hspace{-30pt}
\begin{subfigure}[h]{0.02\textwidth}
\begin{tikzpicture}[auto,
  node distance = 2mm,
  start chain = going below,
  start/.style = {ellipse,draw,text width= 3em,rounded corners,blur shadow,fill=white,
  on chain,align=center},
  box/.style = {draw,rounded corners,blur shadow,fill=white,
  on chain,align=center},
  cond/.style = {diamond,aspect=1,draw,rounded corners,blur shadow,fill=white,
  on chain,align=center},
  n/.style = {draw=none, on chain,align=center}]
  stop/.style = {circle,minimum width=20pt,draw,blur shadow,fill=white, on chain,align=center}]
 \node[n] (l1)    {\nonterm{bsearch}};
 \node[n, below left=2mm and 2mm of l1] (l2)    {l.2};
 \node[n, right=2mm of l2] (l3) {\nonterm{while.3}};
 \node[n, right=2mm of l3] (l18) {l.18};
 \node[n, below left=2mm and 2mm of l3] (l4) {l.4};
 \node[n, below=2mm of l3] (l5) {\nonterm{if.5}};
 \node[n, below left=2mm and 2mm of l5] (l6) {l.6};
 \node[n, below=2mm of l5] (l7) {l.7};
 \node[n, below right=2mm and 2mm of l5] (l8) {\nonterm{if.8}};
 \node[n, below left=2mm and 2mm of l8] (l11) {l.11};
 \node[n, below=2mm of l8] (l12) {\nonterm{if.12}};
 \node[n, below=2mm of l12] (l15) {l.15};
 
\draw [arrow] (l1) -- (l2);
\draw [arrow] (l1) -- (l3);
\draw [arrow] (l1) -- (l18);
\draw [arrow] (l3) -- (l4);
\draw [arrow] (l3) -- (l5);
\draw [arrow] (l5) -- (l6);
\draw [arrow] (l5) -- (l7);
\draw [arrow] (l5) -- (l8);
\draw [arrow] (l8) -- (l11);
\draw [arrow] (l8) -- (l12);
\draw [arrow] (l12) -- (l15);
 %\begin{scope}[rounded corners,-latex]
 % \path (b2.-40) edge[bend left=50] (b4.40)
 % (b1) edge (b2)
 % (b2) edge (b3);
 % \draw (b3.200) -- ++(0,0) -| ([xshift=-5mm]b2.west) |-
 % ([yshift=3mm]b2.130) -- (b2.130);
 %\end{scope}
\end{tikzpicture}
\end{subfigure}
\caption{\<bsearch> source, execution grammar, and an execution tree for \<bsearch(1, [1], 1)>}
\label{fig:bsearch}
\end{figure}
