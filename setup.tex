\section{Experimental Setup} \label{sec:setup}

\subsection{Research Questions}

This study addresses the following research questions:

\begin{itemize}
    \item \textbf{RQ1:} How accurately do statistical estimators of maximum reachability approximate the ground truth obtained using our proposed ensemble measurement approach?
    
    \item \textbf{RQ2:} How does the performance of estimators based on our proposed ground truths compare to their performance using ground truths from prior studies?
    
    \item \textbf{RQ3:} To what extent are the estimates of maximum reachability $S$ sensitive to changes in the sampling unit size $r$?
\end{itemize}

\subsection{Fuzzers and Subject Programs}

For our experiments, we use AFL++ \cite{aflpp}, a widely recognized state-of-the-art greybox fuzzer known for its performance and extensive adoption in recent research, particularly in reachability estimation studies. Using the same fuzzer as prior work enables direct and fair comparisons when evaluating estimator performance (RQ2).

We select a diverse set of real-world C programs from the FuzzBench platform \cite{metzman2021fuzzbench}, which provides a standardized fuzzing benchmarking environment. To ensure comparability with earlier studies (specifically for RQ2), we include all subject programs used in \cite{reachability_2023}, along with several additional targets that are compatible with AFL++ and suitable for large-scale fuzzing.

